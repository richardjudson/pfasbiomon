\nonstopmode{}
\documentclass[a4paper]{book}
\usepackage[times,inconsolata,hyper]{Rd}
\usepackage{makeidx}
\usepackage[utf8]{inputenc} % @SET ENCODING@
% \usepackage{graphicx} % @USE GRAPHICX@
\makeindex{}
\begin{document}
\chapter*{}
\begin{center}
{\textbf{\huge pfasbiomon}}
\par\bigskip{\large \today}
\end{center}
\inputencoding{utf8}
\ifthenelse{\boolean{Rd@use@hyper}}{\hypersetup{pdftitle = {pfasBiomon: PFAS Biomonitoring and In Vitro PODs to generate Margins of Exposure}}}{}
\begin{description}
\raggedright{}
\item[Type]\AsIs{Package}
\item[Title]\AsIs{PFAS Biomonitoring and In Vitro PODs to generate Margins of Exposure}
\item[Version]\AsIs{0.1.0}
\item[Author]\AsIs{Richard Judson}
\item[Maintainer]\AsIs{Richard Judson }\email{judson.richard@epa.gov}\AsIs{}
\item[Description]\AsIs{The package takes data from PFAS biomonitoring studies (blood, plasma)
and in vitro bioactivity data and calculates margins of exposure (MoEs). This 
supports the manuscript titled ``A Comparison of In Vitro and In Vivo Points 
of Departure with Human Biomonitoring Levels for Per- and 
Polyfluoroalkyl Substances (PFAS)''}
\item[License]\AsIs{GPL-3}
\item[Encoding]\AsIs{UTF-8}
\item[LazyData]\AsIs{true}
\item[RoxygenNote]\AsIs{7.2.3}
\end{description}
\Rdcontents{\R{} topics documented:}
\inputencoding{utf8}
\HeaderA{ACToR.parse}{Parse the ACToR files and fill in the remaining missed data}{ACToR.parse}
%
\begin{Description}\relax
Parse the ACToR files and fill in the remaining missed data
\end{Description}
%
\begin{Usage}
\begin{verbatim}
ACToR.parse(dataset = "PFAS_3M")
\end{verbatim}
\end{Usage}
%
\begin{Arguments}
\begin{ldescription}
\item[\code{dir}] The directory where the lists are stored
\end{ldescription}
\end{Arguments}
\inputencoding{utf8}
\HeaderA{ACToR.parse.fillin}{Parse the ACToR files}{ACToR.parse.fillin}
%
\begin{Description}\relax
Parse the ACToR files
\end{Description}
%
\begin{Usage}
\begin{verbatim}
ACToR.parse.fillin(
  datasets = c("Denmark", "Germany", "HealthCanada", "Japan", "Michigan",
    "Norway_Nordic", "Ohio", "Sweden")
)
\end{verbatim}
\end{Usage}
%
\begin{Arguments}
\begin{ldescription}
\item[\code{dir}] The directory where the lists are stored
\end{ldescription}
\end{Arguments}
\inputencoding{utf8}
\HeaderA{dnt.check}{check the processing for the DNT PODs}{dnt.check}
%
\begin{Description}\relax
check the processing for the DNT PODs
\end{Description}
%
\begin{Usage}
\begin{verbatim}
dnt.check(to.file = F)
\end{verbatim}
\end{Usage}
\inputencoding{utf8}
\HeaderA{driver}{Run all of the analyzes starting with the hand-edited data file}{driver}
%
\begin{Description}\relax
Run all of the analyzes starting with the hand-edited data file
\end{Description}
%
\begin{Usage}
\begin{verbatim}
driver()
\end{verbatim}
\end{Usage}
\inputencoding{utf8}
\HeaderA{extract\_NHANES}{Code to format the NHANES data Note that this will only run under R 4.x}{extract.Rul.NHANES}
%
\begin{Description}\relax
Code to format the NHANES data
Note that this will only run under R 4.x
\end{Description}
%
\begin{Usage}
\begin{verbatim}
extract_NHANES(dir = "../data/", dataset = "PFAS_J")
\end{verbatim}
\end{Usage}
%
\begin{Arguments}
\begin{ldescription}
\item[\code{dir}] The directory to look for data

\item[\code{dataset}] The name of the NHANES dataset
\end{ldescription}
\end{Arguments}
\inputencoding{utf8}
\HeaderA{pfas.for.mdh}{Pull the PFAS data fro MDH}{pfas.for.mdh}
%
\begin{Description}\relax
Pull the PFAS data fro MDH
\end{Description}
%
\begin{Usage}
\begin{verbatim}
pfas.for.mdh(to.file = F)
\end{verbatim}
\end{Usage}
\inputencoding{utf8}
\HeaderA{pfasBiomonitoringMerge}{Merge the ACToR data with NHANES and put in a common format}{pfasBiomonitoringMerge}
%
\begin{Description}\relax
Merge the ACToR data with NHANES and put in a common format
\end{Description}
%
\begin{Usage}
\begin{verbatim}
pfasBiomonitoringMerge()
\end{verbatim}
\end{Usage}
\inputencoding{utf8}
\HeaderA{pfasBiomonitoringOldNewMergeForFiltering}{Setup the new filtered input file using the latest assays from Doris}{pfasBiomonitoringOldNewMergeForFiltering}
%
\begin{Description}\relax
Setup the new filtered input file using the latest assays from Doris
\end{Description}
%
\begin{Usage}
\begin{verbatim}
pfasBiomonitoringOldNewMergeForFiltering()
\end{verbatim}
\end{Usage}
\inputencoding{utf8}
\HeaderA{pfasBiomonitoringOutliers}{Find outliers}{pfasBiomonitoringOutliers}
%
\begin{Description}\relax
Find outliers
\end{Description}
%
\begin{Usage}
\begin{verbatim}
pfasBiomonitoringOutliers()
\end{verbatim}
\end{Usage}
\inputencoding{utf8}
\HeaderA{pfasBloodLevelxChainLength}{Plotthe blood levels by chain length}{pfasBloodLevelxChainLength}
%
\begin{Description}\relax
Plotthe blood levels by chain length
\end{Description}
%
\begin{Usage}
\begin{verbatim}
pfasBloodLevelxChainLength(to.file = F)
\end{verbatim}
\end{Usage}
%
\begin{Arguments}
\begin{ldescription}
\item[\code{to.file}] If TRUE, write graphs to a file

\item[\code{data.version}] Label of folder where input data sits
\end{ldescription}
\end{Arguments}
\inputencoding{utf8}
\HeaderA{pfasBloodLevelxChainLengthGG}{Plotthe blood levels by chain length}{pfasBloodLevelxChainLengthGG}
%
\begin{Description}\relax
Plotthe blood levels by chain length
\end{Description}
%
\begin{Usage}
\begin{verbatim}
pfasBloodLevelxChainLengthGG(to.file = F)
\end{verbatim}
\end{Usage}
%
\begin{Arguments}
\begin{ldescription}
\item[\code{to.file}] If TRUE, write graphs to a file

\item[\code{data.version}] Label of folder where input data sits
\end{ldescription}
\end{Arguments}
\inputencoding{utf8}
\HeaderA{pfasChemicalTable}{Generate a table of the chemicals and the data we have for them}{pfasChemicalTable}
%
\begin{Description}\relax
Generate a table of the chemicals and the data we have for them
\end{Description}
%
\begin{Usage}
\begin{verbatim}
pfasChemicalTable()
\end{verbatim}
\end{Usage}
\inputencoding{utf8}
\HeaderA{pfasCorrrectedMOE}{Calculate the TK-corrected MOE values}{pfasCorrrectedMOE}
%
\begin{Description}\relax
Calculate the TK-corrected MOE values
\end{Description}
%
\begin{Usage}
\begin{verbatim}
pfasCorrrectedMOE()
\end{verbatim}
\end{Usage}
\inputencoding{utf8}
\HeaderA{pfasCorrrectedMOE.noTK}{Calculate the TK-corrected MOE values}{pfasCorrrectedMOE.noTK}
%
\begin{Description}\relax
Calculate the TK-corrected MOE values
\end{Description}
%
\begin{Usage}
\begin{verbatim}
pfasCorrrectedMOE.noTK()
\end{verbatim}
\end{Usage}
\inputencoding{utf8}
\HeaderA{pfasHalfLife}{Extract the half-life information from Dawson et al. supplemental file}{pfasHalfLife}
%
\begin{Description}\relax
Extract the half-life information from Dawson et al. supplemental file
\end{Description}
%
\begin{Usage}
\begin{verbatim}
pfasHalfLife(species = "Human", route = "Oral")
\end{verbatim}
\end{Usage}
%
\begin{Arguments}
\begin{ldescription}
\item[\code{species}] The species to use

\item[\code{route}] The exposre route
\end{ldescription}
\end{Arguments}
\inputencoding{utf8}
\HeaderA{pfasInVitroVsRat}{Compare the in vitro PODs vs the Rat POD concentration}{pfasInVitroVsRat}
%
\begin{Description}\relax
Compare the in vitro PODs vs the Rat POD concentration
\end{Description}
%
\begin{Usage}
\begin{verbatim}
pfasInVitroVsRat(to.file = F)
\end{verbatim}
\end{Usage}
%
\begin{Arguments}
\begin{ldescription}
\item[\code{to.file}] If TRUE, write graphs to a file

\item[\code{data.version}] Label of folder where input data sits
\end{ldescription}
\end{Arguments}
\inputencoding{utf8}
\HeaderA{pfasMoeBoxplot}{Plot the distribution of PODs}{pfasMoeBoxplot}
%
\begin{Description}\relax
Plot the distribution of PODs
\end{Description}
%
\begin{Usage}
\begin{verbatim}
pfasMoeBoxplot(to.file = F)
\end{verbatim}
\end{Usage}
%
\begin{Arguments}
\begin{ldescription}
\item[\code{dir}] The directory where the lists are stored
\end{ldescription}
\end{Arguments}
\inputencoding{utf8}
\HeaderA{pfasMoeBoxplotOPPT}{Plot the distribution of PODs}{pfasMoeBoxplotOPPT}
%
\begin{Description}\relax
Plot the distribution of PODs
\end{Description}
%
\begin{Usage}
\begin{verbatim}
pfasMoeBoxplotOPPT(to.file = F)
\end{verbatim}
\end{Usage}
%
\begin{Arguments}
\begin{ldescription}
\item[\code{dir}] The directory where the lists are stored
\end{ldescription}
\end{Arguments}
\inputencoding{utf8}
\HeaderA{pfasPartitionCoefficients}{Get the partition coefficients}{pfasPartitionCoefficients}
%
\begin{Description}\relax
Get the partition coefficients
\end{Description}
%
\begin{Usage}
\begin{verbatim}
pfasPartitionCoefficients(species = "Human")
\end{verbatim}
\end{Usage}
%
\begin{Arguments}
\begin{ldescription}
\item[\code{to.file}] If TRUE, write graphs to a file

\item[\code{data.version}] Label of folder where input data sits
\end{ldescription}
\end{Arguments}
\inputencoding{utf8}
\HeaderA{pfasPerChemicalMoeBoxplot}{Build the raw MoE Boxplots}{pfasPerChemicalMoeBoxplot}
%
\begin{Description}\relax
Build the raw MoE Boxplots
\end{Description}
%
\begin{Usage}
\begin{verbatim}
pfasPerChemicalMoeBoxplot(to.file = F)
\end{verbatim}
\end{Usage}
%
\begin{Arguments}
\begin{ldescription}
\item[\code{to.file}] If TRUE, write graphs to a file
\end{ldescription}
\end{Arguments}
\inputencoding{utf8}
\HeaderA{pfasRawMoeBoxplotsGG}{Build the raw MoE Boxplots}{pfasRawMoeBoxplotsGG}
%
\begin{Description}\relax
Build the raw MoE Boxplots
\end{Description}
%
\begin{Usage}
\begin{verbatim}
pfasRawMoeBoxplotsGG(to.file = F)
\end{verbatim}
\end{Usage}
%
\begin{Arguments}
\begin{ldescription}
\item[\code{to.file}] If TRUE, write graphs to a file
\end{ldescription}
\end{Arguments}
\inputencoding{utf8}
\HeaderA{printCurrentFunction}{Print the name of the current function}{printCurrentFunction}
%
\begin{Description}\relax
Print the name of the current function
\end{Description}
%
\begin{Usage}
\begin{verbatim}
printCurrentFunction(comment.string = NA)
\end{verbatim}
\end{Usage}
%
\begin{Arguments}
\begin{ldescription}
\item[\code{comment.string}] An optinal string to be printed
\end{ldescription}
\end{Arguments}
\printindex{}
\end{document}
